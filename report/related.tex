Analytic power modeling~\cite{Brooks2000, Vijaykrishnan2000, Li2009, Leng2013, Shao2014}
combined with microarchitectural cycle-level software simulators~\cite{Binkert2011, Patel2011, Wenisch2006}
is widely used for computer architecture research. This enables early microarchitecture-level
design space exploration without necessitating RTL development. However, microarchitectural
software simulators run much slower than real systems, making full execution of applications infeasible.
Moreover, for different microarchitectures with new technologies, a new analytic power model
is constructed and strictly validated against real systems or detailed circuit / gate-level simulations,
which is very difficult for non-traditional hardware designs.

Power modeling based on performance-monitoring counters is also
popular for power estimation~\cite{Bellosa2000, Bircher2003, Isci2003,
  Bircher2005, Bircher2007, Bertran2013}.  This method provides a
quick power estimate by profiling full execution of applications.
However, its application is also limited to well-known microarchitectures.
Moreover, it is only practical with existing existing physical systems since
standard microarchitectural software simulators are extremely slow.

There are significant efforts to accelerate power estimation using an FPGA
~\cite{Coburn2005, Ghodrat2007, Atienza2006, Bhattacharjee2008, Sunwoo2010, Yang2015, Kim2016}.
Coburn et al.~\cite{Coburn2005} implement detailed power models directly on the FPGA,
which suffers from large FPGA resource overhead. Ghodrat et al.~\cite{Ghodrat2007}
expands the ideas of Coburn et al.~\cite{Coburn2005} by translating part of 
power computation logic on the FPGA to software to reduce FPGA resource overhead,
but introduces communication overhead between FPGA and software. These methods
are inherently lack of scalability due to large overhead of power computation logic. 

On the other hand, Atienza et al.~\cite{Atienza2006} manually attached activity monitoring units
to the target design on the FPGA. Bhattacharjee et al.~\cite{Bhattacharjee2008} manually implement
event counters on top of RTL designs to estimate power dissipation from FPGA emulation.
PrEsto~\cite{Sunwoo2010} generates power models from signals manually selected by users.
In addition, This technique also requires additional manual efforts to instrument existing
FPGA simulators with power models. These methods have less FPGA resource overhead,
but require significant manual efforts as well as designers' intuition.

Yang et al.~\cite{Yang2015} proposes to construct cycle-by-cycle power models
by pruning probed signals with machine learning without requiring designers' intuition.
However, this method simply assumes dynamic power dissipation is a linear function of
primary registers' toggle activities, which is not necessarily always true.
Also, it instruments the RTL designs with the power computing logic as in PrEsto~\cite{Sunwoo2010},
which may be automated by commercial CAD tools.

Strober~\cite{Kim2016} automatically transforms RTL designs into FPGA performance simulators
and also automatically instrument the simulators with scan chains to capture RTL state snapshots.
Then, the RTL state snapshots are randomly sampled from the FPGA simulation and replayed on
detailed gate-level simulation to compute the average power. Although, Strober quickly provides
accurate power estimates with statistically bounded errors, the average power should be computed 
after the FPGA simulation finishes, preventing online power computation, which is the motivation
of this paper.