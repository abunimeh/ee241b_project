Power dissipation has been a major design constraint not only for embedded and portable systems
but also for high-end servers and datacenters. A plethora of design strategies have been suggested
to meet the power constraint while achieving maximum performance. Thus, power estimation 
methodologies are very important for the evaluation and the validation of novel hardware designs.
However, accurate and fast application-specific power estimation for complex hardware design
is still a big challenge for system designers and application developers, which will play
a more crucial role in the near future as Moore's law is near the end.

Microarchitecture-level power analysis tools calibrated against representative hardware designs
are widely used by computer architects~\cite{Brooks2000, Vijaykrishnan2000, Li2009, Leng2013, Shao2014}.
These power models need activity statistics driven from microarchitectural cycle-level software
simulators~\cite{Binkert2011, Patel2011, Wenisch2006}. 
This approach helps designers study trace-offs with design parameters in early design phase
without developing RTL designs. However, this methodology is limited to designs that are similar 
to what the abstract model is build upon, and requires long simulation times to collect 
microarchitectural activities from microarchitectural software simulators. 
Moreover, it is very hard to apply this methodology to non-traditional hardware designs
such as application-specific accelerators as their power models should be validated against
RTL designs or existing systems.

Once RTL implementations are available, energy efficiency as well as cycle time and area
can be evaluated using commercial CAD tools. The existing CAD tools provide very accurate
performance and power estimates. However, their simulation time is extremely slow, preventing
design space exporation for realistic applications running on complex hardware designs.

This paper describes an automatic activity-based power modeling methodology for arbitrary RTL,
which enables fast and accurate power estimation for any RTL designs.
First, signals, which determines the cycle behaviors of the target design, are automatically
selected by analyzing the circuit graph. Next, a design-specific activity power model is constructed and
trained in terms of the selected signal activities. In addition, the target RTL design is automatically
instrumented so that the signal activities can be probed during performance simulation on the FPGA.
Finally, power dissipation is estimated in real time by pausing the FPGA performance simulation and
reading out the signal activities from the FPGA.